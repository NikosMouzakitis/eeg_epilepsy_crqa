\documentclass{article}

\usepackage[utf8]{inputenc}
\usepackage{geometry}
\usepackage{booktabs} 
\usepackage{comment} 
\usepackage{amsmath}
\usepackage{amssymb}
\usepackage{bm}
\usepackage{algorithm}
\usepackage{algpseudocode}
\usepackage{siunitx}
\usepackage{caption}
\usepackage{subcaption}
\usepackage{graphicx}
\usepackage{float}
\usepackage{array}
\usepackage[numbers,sort]{natbib} 
\usepackage{multirow}       % For table multirows
\usepackage[font=small,labelfont=bf]{caption} % For caption formatting

% Preamble: Define title, author, date
\title{Epilepsy Detection from Multi-Channel EEG Using Cross-Recurrence Quantification Analysis and Machine Learning}
\author{Nikolaos Mouzakitis} % Use \\ for a line break
\date{\today} % You can use a specific date or \today for the current date
\begin{document}
	\maketitle

	\section*{Introduction}
	Epilepsy is a neurological disorder, 
	which generates recurrent, unprovoked seizures. 
	Seizures are the result of abnormal neuronal brain 
	activity which subsequently lead into 
	disturbances in the behavior, sensation, the consciousness and the movement 
	of the affected subject.
	Information gathered by World Health Organization (WHO), 
	report that epilepsy affects 
	around 50 million people worldwide, 
	placing it as one of the most common 
	neurological conditions globally.

	Epilepsy has its origin in diverse and various causes depending on many factors. 
	The generation of epileptic activity can be an impact of identifiable structural, 
	genetic, infectious, metabolic, or immune-related abnormalities, 
	while for almost 50\% of the cases, their trigger still remains unknown\cite{causes}. 
	Depending on the brain regions involved, a seizure can have several classifications. 
	It can be classified as focal (if its origin is a specific area) 
	or a generalized one (involving both hemispheres). 
	Clinically, seizures are characterized by high variability, 
	from short term lapses in awareness, up to 
	convulsive episodes, and the fact of their unpredictable occurrence 
	has severe impact in patient’s quality of life.

	For the diagnosis and monitoring of epilepsy, 
	electroencephalography (EEG) 
	is widely utilized, as is offers a non invasive 
	technique for recording brain’s electrical 
	activity using surface electrodes. 
	EEG signals contain temporal 
	information that reflects 
	the dynamic interactions 
	of neuronal populations. 
	During the seizure events, 
	the presence of characteristic patterns 
	such as spikes, sharp waves, or rhythmic 
	discharges often appear, 
	distinguishing this kind of activity from the normal 
	background rhythms. 
	As a result, EEG analysis holds
	a central role both in clinical diagnosis 
	and for research related to this application domain.

	Recent advances in signal processing combined with
	machine learning have greatly 
	improved the ability EEG data analysis. 
	Techniques like time-frequency decomposition, 
	nonlinear dynamics, recurrence analysis, 
	and deep neural networks 
	can offer new solutions for automated extraction 
	of complex spatial and temporal 
	features from EEG recordings. 
	The ultimate goal of these approaches is
	to support clinicians by providing objective, 
	data-driven tools for 
	seizure detection, prediction, and classification, 
	in order to contribute to better patient care and personal treatments.

	In summary, epilepsy represents a major public health challenge due to its prevalence, 
	variability, and social consequences. Understanding the electrophysiological mechanisms 
	that drive seizure generation and developing reliable methods for automatic EEG analysis 
	remains a crucial research direction in modern neuroscience and biomedical engineering.

			%small intro and foundational works to have as introduction.test papers.
	Recurrence Quantification Analysis (RQA) and Cross-Recurrence Quantification Analysis (CRQA) are
	nonlinear methods for the analysis of nonstationary time series, such as EEG signals. 
	The offer the quantification of the recurring patterns in phase space trajectories \cite{trulla1996, webber2005}. 
	Introduced by Trulla et al.\cite{trulla1996} (directly built on quantifying recurrence plots\cite{eckmann1987}) 
	and expanded by Webber and Zbilut\cite{webber2005}, RQA measures metrics 
	like recurrence rate, determinism, and laminarity to capture dynamic system behavior. 
	Thomasson et al.\cite{thomasson2002} in their work, demonstrated RQA’s applicability on EEG data, mentioning 
	the robustness it shows in accordance to noise
	and nonstationarity. Marwan et al.\cite{marwan2013} further advanced recurrence plot techniques,
	emphasizing on developing a confidence measure of RQA in detecting dynamic transitions.
	Works like these, serve as a foundation of applying RQA and CRQA on EEG 
	studies in various conditions such as epilepsy, cognitive disorders and others. XXX

	\section{Related Work}
			%ok, eeg OK
			Frolov et al.\cite{frolov} proposed an approach to analyze frequency based multiplex brain networks
			using recurrence quantification analysis (RQA) 
			on EEG data, and demonstrated the way that recurrence-based 
			synchronization indices can effectively capture 
			both within-frequency (intralayer) and cross-frequency (interlayer) 
			functional connectivity during cognitive tasks. 
			Their work showed that RQA is particularly suitable for analyzing 
			non-stationary EEG signals and revealed
			important insights about the evolution of functional connectivity 
			patterns during cognitive tasks. In addition the dataset
			used in this research are openly available in a Figshare repository.

			%ok.   eeg,Alzheimer OK
			Núñez et al. \cite{nunez2020characterization} worked with 
			resting-state EEG recordings from subjects with mild cognitive impairment(MCI), 
			Alzheimer's disease(AD), and healthy ground truth controls in order to detect 
			frequency based changes into their brain dynamics. 
			By blending wavelet based Kullback–Leibler divergence
			(KLD) for capturing non-stationarity,
			and two RQA
			metrics(\textit{entropy of the recurrence point density}
			and the \textit{median of the recurrence point density}) insights have been
			extracted related to neurodegeneration presence.
			Research's findings show that MCI and AD are presenting notable changes in 
			the recurrence structure and non-stationarity of EEG signals,
			and more specifics on the theta and beta frequency bands.
			Therefore, recurrence based dynamics show a capability as potential 
			biomarkers for monitoring and detecting early Alzheimer's disease and its progression.

			%% EEG, RQA-CRQA, MCI, OKAY.
			MCI has also investigated by Timothy et al.\cite{timothy2017classification}, where 
			researchers have focused on 
			the classification of MCI using EEG signals and 
			combining RQA and CRQA methods. Analysis has been performed on both resting-state 
			(eyes closed) and task-based (short-term memory) EEG data, 
			focusing on complexity (via RQA) and synchronization (via CRQA) features. 
			Their results indicate that MCI patients exhibit lower complexity
			and higher inter- and intra-hemispheric synchronization compared to healthy controls, 
			particularly during memory tasks. 
			The study also proposes a novel feature space approach using RQA and CRQA measures, 
			achieving high classification accuracy (91.7\%) under task conditions. 
				


			%epilepsy-ok eeg   OK
			Fan and Chou \cite{fan2019detecting} have also proposed 
			an approach for real-time epileptic seizure detection
			using as a method the analysis of temporal synchronization 
			patterns of EEG signals with recurrence networks and spectral graph theory. 
			Recurrence plots were used for the modeling of the EEG dynamics, 
			extracting graph theory's features for quantifying the synchronization. 
			Results showed high sensitivity of 98.48\% and low latency
			(6 seconds) for detecting seizure on the CHB-MIT dataset, 
			performing better than other RQA measures.  

			%ok, EEG, RQA, ASD    OK
			Heunis and co-authors\cite{heunis2018} have utilized resting state EEG and RQA in order to
			distinguish individuals of ages 0-18 of two categories; ASD(autism spectrum disorder) and typically developing.
			RQA features were extracted and tested on various linear and nonlinear classifiers achieving 92.9\% classification
			accuracy with nonlinear SVM classifier.



			% EEG, aging aisthitiriako-kinitiko systima. ok.  OK
			Author in \cite{pitsik}, investigated changes related to aging in 
			brain sensorimotor systems using 
			RQA and theta-band functional connectivity in EEG signals. 
			In the study a VR experimental paradigm was 
			utilized with auditory stimulus across different age groups(young and elder subjects). 
			Key findings include that elder subjects present 
			decreased EEG complexity during motor preparation stages as 
			measured by RQA metrics (\textit{$\Delta$RR and $\Delta$RTE}), 
			and had increased theta band functional connectivity 
			highlighting the potential of RQA in detecting 
			age related biomarkers that were not detectable using 
			standalone signal spectral analysis.

			%cognitive, eeg OK.    OK
			Guglielmo et al. \cite{guglielmo}  
			utilized RQA features extracted 
			by EEG signals for the purpose of classification
			of cognitive performance during mental arithmetic tasks. 
			They used frontal and parietal EEG signals 
			and analyzed them, from 36 participants by extracting 
			six RQA metrics (\textit{recurrence rate, determinism, 
			laminarity, entropy, maximum diagonal line length and average diagonal line length}) 
			from four electrodes (F7, Pz, P4, Fp1). 
			Afterwards by applying machine learning classifiers 
			(SVM, Random Forest, and Gradient Boosting) and
			they reached accuracy of classification above 0.85, 
			showing the potential that RQA holds for 
			generalizing on nonlinear dynamics.
			
			Mihajlović \cite{mihajlovic19} studied the discriminative effiency of traditional spectral features 
			in comparisson to RQA-derived nonlinear metrics for the cognitive effort classification purposes. 
			Utilizing a 4-channel wearable EEG headset, data was recorded while subjects perform tasks
			having variable cognitive load such as relaxation, math, reading. 
			The key finding was that while spectral features alone often yielded higher classification accuracy, 
			RQA features such \textit{recurrence rate,determinism ratio} were consistently ranked among the 
			most important features for discrimination task. A conjuction of a hybrid model using both spectral and RQA features 
			achieved the best overall performance, showing the complementary nature of the methods in brain dynamics exploration. 


			%% epilepsy, CRQA,RQA,sEEG   OK
			Yang and co-authors \cite{yang2019dynamical}, examined stereo electroencephalography (sEEG) 
			recordings of 10 patients with refractory focal epilepsy for analyzing dynamical differences 
			among discreet epileptic phases/states (inter-ictal, pre-ictal, and ictal) and regions. 
			Using recurrence plots and CRQA, they identified epileptogenic channels with longer diagonal
			structures in RPs, which is a sign of more deterministic and recurrent dynamics. 
			Their findings point out that the synchronization among the epileptogenic channels strengthened 
			while seizures events occur, suggesting that these regions dominate the 
			network's dynamics.


			%epilepsy, sEEG , RQA ok
			Lopes et al. \cite{lopes} have proposed a 
			combinatorial framework 
			by mixing RQA with dynamic functional network (dFN) analysis,
			applying it to both MEG and stereo EEG data. 
			The methodology they described is split 
			into five steps: data segmentation, 
			functional network inference, distance computation alongside networks, 
			recurrence plot construction and finally RQA. 
			The study demonstrated that functional networks in epilepsy 
			patients recur more quickly than in healthy controls, suggesting RQA on
			dFNs could play the role of a potential biomarker.
			For the EEG dataset investigation, they have showed that the pre-ictal 
			networks shown higher recurrence rates 
			than post-ictal periods, with the $\tau$-recurrence rate ($RR_{\tau}$) proving particularly 
			effective for seizure detection.
			
			%eeg, epilepsy, RQA, OK
			Rangaprakash~\cite{rangaprakash2014} have proposed an application of RQA for the study of
			brain connectivity using multichannel EEG signals. In its work,
			a new CRQA-based feature was proposed (Correlation between 
			Probabilities of Recurrence (CPR)), a nonlinear and non-parametric 
			phase synchronization technique. Afterwards it was utilized for the analysis 
			of functional connectivity in epilepsy subjects during eyes-open/eyes-closed conditions.
			The results demonstrated that CPR outperformed other known traditional 
			linear methods on distinguishing seizure and pre-seizure states, 
			identifying epileptic foci, and differentiating alongside eyes-open and eyes-closed conditions. 

			%eeg-epilepsy   OK.
			In another study which demonstrates the effectiveness of RQA in 
			analyzing EEG signals for epilepsy detection,
			Gruszczyńska et al.\cite{gruszczynska2019} applied RQA on such signals
			in order to distinguish epileptic from healthy patients using recordings 
			from frontal and temporal lobe electrodes (Fp1, Fp2, T3, T4). 
			In their findings they have showed that the epileptic signals present more periodic
			dynamics in comparison to healthy controls, by producing higher values of 
			RQA parameters such as determinism,
			laminarity, and longest diagonal line. The combination of RQA with
			Principal Component Analysis for dimensionality reduction and visualization, achieved 86.8\% 
			classification accuracy with SVM. Authors also demonstrated RQA's capability
			to identify pathological patterns in EEG signals without the 
			requirement of seizure events during recording which have bad impact on the subject's health.



			%sEEG
			Another study utilizing advanced nonlinear analysis techniques for neural correlation investigation to
			cognitive functions \cite{mo} used \textit{stereoelectroencephalography (sEEG)} combined alongside RQA 
			for the examination of the relationship of the DMN and empathy. 
			Correlations have been detected relating specific RQA metrics 
			(mean diagonal line length, entropy of diagonal line lengths, trapping time) 
			and empathy scores, particularly within DMN subsystems. 

			%epilepsy EEG
			Regarding epilepsy diagnosis, authors in \cite{palanisamy2024} proposed a new framework 
			utilizing the combintation of RQA with genetic algorithms and Bayesian classifiers for 
			identifying corresponding biomarkers for seizure detection. 
			They utilized five distance norms (e.g., Euclidean, Mahalanobis) and multiple thresholds 
			for extracting recurrence features from EEG signals, achieving 100\% classification accuracy. 
			More specific, the \textit{transitivity} feature has shown capability of a highly discriminative biomarker, 
			performing better compared to traditional linear methods. 

			%epilepsy EEG
			Ngamga et al.\cite{ngamga2016} studied the performance achieved of RQA and Recurrence Network (RN) measures in identifying 
			pre-seizure states from multi-day, multi-channel intracranial EEG (iEEG) 
			recordings of epilepsy patients. 
			Results highlighted the correlation among RQA measures (determinism, laminarity, and mean recurrence time) in 
			detecting seizure precursors, while RN measures (average shortest path length and network transitivity) provided 
			complementary but not so consistent insights than using the application of RQA measures alone.

			Gao et al.\cite{gao2020automatic} examined the application of RQA 
			in the domain of automated epilepsy detection. 
			Authors utilized a hybrid scheme combining nonlinear features(related to Approximate Entropy(ApEn) and RQA metrics) 
			from the publicly available Bonn EEG dataset\cite{bonndataset} with a deep learning classifier.
			Their key finding was that while ApEn and RQA features alone could 
			achieve good classification accuracy, 
			their performance was increased when used as input 
			features for a Convolutional Neural Network (CNN). By constructing this hybrid approach,
			classification accuracy rised on 99.26\% for distinguishing ictal from inter-ictal 
			and healthy EEG signals, demonstrating the potential of the synergy among traditional metrics
			and modern deep learning architectures.





			%%%	studies with fmri

			% Alzheimer, fmri, ok   OK
			Researchers in \cite{rezaei},
			have applied RQA on resting-state fMRI data from 
			TgF344-AD rats(a transgenic rat model which will eventually develop Alzheimer’s disease)
			and their healthy-control counterparts wild-type rats(WT),
			in order to detect early stage biomarkers for the disease.
			By analyzing Default Mode-Like Network (DMLN) 
			using RQA metrics(\textit{entropy, recurrence rate, determinism 
			and average diagonal line length}) 
			changes have been detected in regions of 
			the basal forebrain, hippocampal fields (CA1, CA3), and visual 
			cortices (V1, V2). Also on the study's findings include reduced predictability in 
			WT rats with aging, while AD rats exhibited less decline
			in predictability, suggesting some unknown yet countereacting mechanisms. 
			This study highlights RQA's sensitivity for nonlinear dynamics 
			in preclinical AD and the code used is also publicly available.


			%schizophrenia, fmri, ok  OK
			Lombardi et al.\cite{Lombardi2014} investigated the 
			nonlinear properties in fMRI BOLD signals 
			during a working memory task in 
			schizophrenic patients and healthy controls. 
			They have attempted by using RQA, to analyze recurrence plots 
			for quantifying determinism, trapping time, 
			and maximal vertical line length 
			in functionally relevant brain clusters. 
			Outcome revealed differences in 
			the dynamics between the two groups, 
			and more specific in working memory and DMN areas. 
			While their work have focused on fMRI, the methodology can be adapted also into
			EEG signals, which can offer a higher resolution for capturing rapid neural dynamics.

			%ok. schizophrenia, fMRI  OK
			Kang et al. \cite{kang}, in their study explore the dynamics and functional connectivity of the 
			Default Mode Network (DMN) in schizophrenia, applying RQA-CRQA on resting-state fMRI data. 
			Findings include decreased \textit{determinism} between specific DMN regions 
			(vMPFC-posterios cingulate and vMPFC-precuneus) in first-episode schizophrenia patients, 
			as a signal of disturbed predictability of functional interactions. 
			Moreover, their results achieve to correctly classify using SVM(support vector machine)
			schizophrenia patients from healthy controls with 77\% classification accuracy.


			%npsle
			In their research, Pentari et al.\cite{pentari22} have applied CRQA to resting-state fMRI data 
			for examining the dynamic functional connectivity on patients with neuropsychiatric systemic 
			lupus erythematosus (NPSLE). Results contain the fact that CRQA metrics, such as determinism,
			appear more sensitive than conventional static functional connectivity methods in order to
			identify aberrant connectivity patterns that correlated with visuomotor performance. 
			The study focused on 16 frontoparietal regions and found that CRQA could detect 
			both increased and decreased connectivity in NPSLE patients compared against the healthy controls. 
			Building on these findings, Pentari et al.\cite{pentari23} subsequently expanded 
			the investigation to whole brain network analysis in a larger cohort. 
			In this study they demonstrate the capability of CRQA to integrate multiple recurrence metrics 
			for revealing both hyperconnectivity in parietal regions (angular gyrus and superior parietal lobule) 
			and hypoconnectivity in medial temporal structures (hippocampus and amygdala). 
		%	Notably, the dynamic connectivity measures showed stronger associations with cognitive 
		%	performance than structural measures, particularly for verbal episodic memory. 

			%% modeling, RQA, ok   OK
			In addition there have been works where simulated data 
			have been used in conjunction with RQA.
			Lameu et al.\cite{lameu2018}, investigated burst phase synchronization in neural networks using RQA. 
			They analyzed two network types; a small-world network and a network of networks 
			(to mimic better the real human brain), using coupled Rulkov maps to model bursting neurons. 
			By applying RQA, they identified synchronized neuron groups and quantified their 
			sizes during synchronization transitions. The study showed that RQA measures 
			(\textit{recurrence rate, laminarity inspired}(custom feature)\textit{, and average structure size}) complement 
			traditional order parameters by revealing localized synchronization patterns, 
			such as the formation and growth of synchronized clusters.
			Kashyap and Keilholz\cite{Kashyap2019} conducted a comprehensive comparison 
			between simulated brain network models (BNMs) and real rs-fMRI data using 
			dynamic analysis techniques, including Recurrence Quantification Analysis (RQA). 
			In the study they employed two BNMs, the Kuramoto oscillator model and the Firing Rate model, for simulating
			the whole-brain activity, which was then compared to human rs-fMRI data. 
			Among the compared dynamic analysis methods, RQA was proved particularly effective 
			in distinguishing between the models and empirical data, demonstrating that RQA metrics 
			(\textit{recurrence rate, entropy, and average diagonal length}) could robustly separate the empirical data from simulations. 
		   
			
			Shalbaf et al. \cite{shalbaf2014frontal} investigated the synchronization of EEG signals 
			between frontal and temporal regions during propofol anesthesia 
			using \textit{Order Patterns Cross Recurrence Analysis} (OPCR). 
			Their study introduced a novel index, \textit{Order Pattern Laminarity} (OPL), for the quantification of
			neuronal synchronization and compared its performance with the traditional Bispectral Index (BIS). 
			The results demonstrated that OPL correlated more strongly with propofol concentration 
			($P_k = 0.9$) and exhibited faster response times to transient changes in consciousness 
			compared to BIS. Additionally, OPL showed lower variability at the point of loss of 
			consciousness (LOC), suggesting its robustness as a measure of anesthetic depth. 
			This work highlights the potential of recurrence-based methods (e.g., CRQA) 
			for analyzing brain network dynamics under anesthesia, 
			particularly in noisy, non-stationary EEG data.






			



		\begin{table}[h]
		\centering
		\caption{Comparison among the retrieved studies using recurrence analysis}
		\label{tab:comparison}
		\begin{tabular}{@{}lcccc@{}}
		\toprule
		\# & Reference & Modality & Analysis Methods & Network Type \\
		\midrule

		1  & Frolov et al. (2020) & EEG & RQA, CRQA & Multiplex functional networks \\
		2  & Kang el al. (2023) & fMRI & RQA, CRQA & DMN, schizophrenia \\
		3  & Rezaei el al. (2023) & fMRI & RQA & Default model-like network, AD \\
		4  & Lameu et al. (2018) & --- & RQA & Small-world \& cluster network \\
		5  & Lombardi et al. (2014) & fMRI & RQA & schizophrenia,working memory \\
		6  & Pitsik E. (2025) & EEG & RQA & aging \\
		7  & Guglielmo et al. (2022) & EEG & RQA & cognitive tasks \\
		8  & Lopes et al. (2020) & sEEG, MEG & RQA & epilepsy \\
		9  & Pentari et al. (2022) & fMRI & RQA, CRQA & NPSLE \\
		10 & Pentari et al. (2023) & fMRI & CRQA & NPSLE  \\
		11 & Gruszczyńska et al. (2019) & EEG & RQA & epilepsy \\
		12 & Mo et al. (2022) & sEEG & RQA & DMN, epilepsy \\
		13 & Palanisamy et al. (2024) & EEG & RQA & epilepsy \\
		14 & Ngamga et al. (2016) & EEG & RQA,RN & epilepsy \\
		15 & Fan and Chou (2019) & EEG & RQA,RN & epilepsy, seizure detection \\
		16 & Nunez et al. (2020) & EEG & RQA & AD \\
		17 & Yang et al. (2019) & sEEG & RQA,CRQA & epilepsy \\
		18 & Rangaprakash (2014) & EEG & CPR(CRQA-based) & epilepsy \\
		19 & Heunis et al. (2018) & rsEEG & RQA & autism spectrum disorder \\
		20 & Timothy et al. (2017) & EEG & RQA-CRQA & MCI \\
		21 & Kashyap et al. (2019) & fMRI & RQA & distinguish BNMs \\
		22 & Shalbaf et al. (2014) & EEG & CRQA(OPL) &  Anesthesia depth monitoring\\
		23 & Mihajlović. (2019) & EEG & RQA &  cognitive tasks\\

		\bottomrule
		\end{tabular}
		\end{table}
	
				
			%% PATENTS
			\subsection{RQA relevant patents utilizing EEG modality}
			\label{sec:rqa-patents}

			The application of RQA utilizing EEG modality on the biomedical field,
			shows an increasing interest
			not only in academic research, but also in 
			commercial and clinical applications, 
			as we can inspect on recent patent filings. 
			Reviewing these documents can reveal 
			industrial viable solutions 
			being developed for real-time, embedded systems. 
				
			Becker et al.\cite{patcoma} describes on patent (US20080234597A1) 
			a monitoring device and method for creating an assessment of the 
			depth of anesthesia or coma characterizing an individual subject.
			Authors analyzes neuronal EEG data and uses RQA 
			to compute a complexity parameter that 
			quantitatively reflects the level of consciousness. 
			In the device's core, a buffer is utilized for storing 
			time-series data 
			and an analysis circuit performs RQA by reconstructing 
			phase-space trajectories, calculating recurrence plots, 
			and extracting determinism-based complexity measures. 
			This makes possible monitoring the depth of anesthesia level in real-time
			and can be utilized in clinical applications for 
			anesthesia control during surgery or even in long term coma assessment.

			Patent US20250195894A1 \cite{pat2025}, entitled 
			``Systems and Methods for Seizure 
			Detection and Closed-Loop Neurostimulation,'' 
%			provides a view of the current challenges and proposes a solution for 
%			implementing RQA in a resource-constrained environment.
			Inventors proceed in an alternative calculation
			of RQA measures which entirely bypass the construction 
			of the recurrence plot matrix(RP).
			They achieve this, by not creating the traditional RP in order 
			to extract certain metrics from, but by
			calculating them, \textit{on-the-fly}; dynamically accumulating 
			the lengths of diagonal lines as each new 
			data point is processed.
		%	The core innovation involves a concept shift: the rows of the hypothetical 
		%	matrix are aligned such that the main diagonal becomes vertical. 
		%	As each new row of this ``imagined'' matrix is considered 
		%	(f.e, as each new data point is acquired), the algorithm checks for recurrence. 
		%	If a recurrence is found, a counter for the corresponding vertical column 
		%	(representing a diagonal line in the original RP) is incremented. 
		%	If a recurrence ends (f.e, a zero follows a one), the final length of the chain 
		%	is recorded in a histogram of line lengths, and the counter is reset. 
		%	This process directly builds the histogram of diagonal line lengths, 
		%	from which standard RQA metrics like DET, ENTR, and L can be immediately derived, 
		%	without ever storing the full RP matrix.
			This method offers the following advantages:
					\begin{enumerate}
					    \item \textbf{Memory Efficiency:} It does not require the large $N \times N$ comparison matrix, reducing memory usage by approximately 88\%.
					    \item \textbf{Computational Efficiency:} It avoids the expensive read/write cycles 
						    associated with managing the large matrix, reducing processing time by approximately 30\% per channel.
					\end{enumerate}
	
				
			In addition, the patent by \cite{pat2018}, titled 
			''An EEG signal classification model based on genetic algorithm and random forest'',
			presents a framework for EEG signal classification. 
			The inventors propose a hybrid model consisting of three key stages:

				\begin{enumerate}
				    \item \textbf{Feature Extraction:} The method employs a multi-modal feature extraction strategy. 
					    Among other features, it explicitly includes \textit{RQA} metrics from the EEG signal, 
					    alongside other traditional time-domain/frequency features.
				    \item \textbf{Feature Optimization:} A genetic algorithm (GA) is then utilized for feature selection. 
					    Inventors use binary encoding for representing chromosomes, where each bit corresponds to the selection (1) or rejection (0) of a specific 
					    feature from the large extracted pool. The aim of this procedure is to optimize the feature subset in order
					    to have maximum discriminative power.
				    \item \textbf{Classification:} The optimized feature subset is fed into a Random Forest 
					    classifier for a final prediction. 
				\end{enumerate}

			The patent claims that this integrated approach, 
			validated on a public dataset, yields a superior classification 
			accuracy compared to existing methods at the time of filing, 
			while also demonstrating robustness through cross-validation.
		
				%%%% on sleep
			Another patent\cite{pat2019} (CN106512206B) describes an 
			implantable closed-loop deep brain stimulation (DBS) system that uses 
			electrophysiological signals (specific deep brain local field potentials (LFPs) and ECG  signals) for monitoring 
			the states of a human sleeping and adjusting various stimulation parameters in real time.
			A device acquires ECG and deep brain signals for feature extraction 
			in the domains of time and frequency, while 
			calculating complexity measures. 
			RQA metrics such recurrence rate, determinism, entropy and laminarity 
			are utilized alongside other complexity 
			and spectral features to classify sleep states and trigger appropriate stimulation responses.
			Features are then utilized for detection of sleep stages 
			and for emergency detection/alerts (f.e, cardiac arrest or abnormal excitation).
	%		Also the described device adjusts or turns off the stimulation based on the patient's sleep state in order to save battery life and reduce side effects.



		\section{The CHB-MIT EEG Database}
		
		The \textbf{CHB-MIT Scalp EEG Database}\cite{chbmitDataset} is a public collection of EEG recordings 
		from pediatric subjects that includes intractable seizures.
		All the recordings of the database were collected at 
		the Children's Hospital Boston.
		The dataset contains multiple cases (patients), each with long-term scalp 
		EEG signals recorded using the international 10–20 system.
		Table~\ref{tab:chbmit_demographics} provides a summary of the demographic information 
		of the database's subjects.
		This particular database is used in many studies for testing 
		algorithms on epileptic seizure 
		detection and epileptic research in general.
		
		\section{Dataset Description}

		\begin{table}[h!]
		\centering
		\caption{Demographic information for patients in the CHB-MIT Scalp EEG Database.}
		\label{tab:chbmit_demographics}
		\begin{tabular}{|c|c|c|}
		\hline
		\textbf{Case} & \textbf{Gender} & \textbf{Age (years)} \\ \hline
		chb01 & F & 11 \\ \hline
		chb02 & M & 11 \\ \hline
		chb03 & F & 14 \\ \hline
		chb04 & M & 22 \\ \hline
		chb05 & F & 7 \\ \hline
		chb06 & F & 1.5 \\ \hline
		chb07 & F & 14.5 \\ \hline
		chb08 & M & 3.5 \\ \hline
		chb09 & F & 10 \\ \hline
		chb10 & M & 3 \\ \hline
		chb11 & F & 12 \\ \hline
		chb12 & F & 2 \\ \hline
		chb13 & F & 3 \\ \hline
		chb14 & F & 9 \\ \hline
		chb15 & M & 16 \\ \hline
		chb16 & F & 7 \\ \hline
		chb17 & F & 12 \\ \hline
		chb18 & F & 18 \\ \hline
		chb19 & F & 19 \\ \hline
		chb20 & F & 6 \\ \hline
		chb21 & F & 13 \\ \hline
		chb22 & F & 9 \\ \hline
		chb23 & F & 6 \\ \hline
		\end{tabular}
		\end{table}

		\noindent
		The database includes recordings from male and female patients, 
		with their ages in the range of 1.5 to 22 years old.
		Most subjects are children, a fact reflecting the pediatric nature 
		of the dataset.
		This demographic diversity provides a representative sample for 
		studying epileptic activity across different developmental stages.

		The total recording duration spans approximately 982 hours, 
		segmented into 664 different EDF files, 
		where each recording contains 1 hour of data (though durations vary from 1 to over 4 hours). 
		EEG signals were recorded using 23 scalp electrodes 
		(according to the international 10--20 system), 
		but in some files there are extra channels records, 
		such as ECG and EMG references. 
		The sampling rate of the recordings is 
		256 Hz with 16-bit resolution. 
		Seizure events are annotated by domain experts, 
		providing a time-stamp for seizure onsets and offsets. 

			\begin{figure}[h]
			    \centering
			    \includegraphics[width=1.1\textwidth]{edf_rec.png}
			    \caption{EEG recording visualized utilizing Python-MNE, for a patient of the dataset.}
			    \label{}
			\end{figure}



		\newpage
		\section{Filtering}

			Signals of electroencephalogram, in most cases 
			carry not only the desired signal but also added noise 
			and artifacts from physiological (eye blinks, muscle or cardiac activity) 
			and non-physiological sources (f.e, powerline interference). 
			In general, artifacts in a recording consist of all the non-neural signals 
			that are mixed with the pure EEG data. 
			
			Preprocessing is required in order to increase 
			the quality of the signal(signal-to-noise ratio),
			to boost the performance of 
			further examintations in  
			later pipelines in domains 
			like brain-computer interfaces (BCIs) 
			or clinical diagnostics\cite{dhanaseivam2023,jiang2019,sen2023,removeArtifactsReview}.

			%\cite{dhanaseivam2023}\cite{jiang2019}\cite{sen2023}\cite{removeArtifactsReview}.
			
			Some of the common preprocessing techniques are:  
			\begin{itemize}  
			    \item \textbf{Filtering} (f.e, Butterworth, Chebyshev) to remove unwanted frequency bands.  
			    \item \textbf{Regression methods} used for removing ocular artifacts with help of reference channels.  
			    \item \textbf{Blind Source Separation (BSS)} (f.e, ICA, CCA), decompose and isolate neural activity from artifacts.  
			    \item \textbf{Wavelet/EMD-based methods} for non-stationary artifact removal.  
			\end{itemize}  
			
			Also hybrid approaches (f.e, wavelet-ICA) exist, combining multiple techniques for 
			improved artifact rejection. 
			The choice of method is related on 
			computational constraints, artifact type, 
			and the satisfaction of real-time processing needs, if any.

			Effective preprocessing is a key, that ensures reliable feature extraction for later analysis.  

			Among these, wavelet-based methods have have been proposed and used 
			for effective EEG denoising based on the non stationary nature of brain signals.
			Transient artifacts appearing in EEGs in varying frequency patterns 
			are the reason that 
			traditional linear filtering methods struggle to handle them 
			without introducing distortions.
			Wavelet transforms on the other hand, decompose the signal 
			into time-frequency representations using scalable, 
			localized basis functions called wavelets, 
			enabling the analysis: the signal is broken down into approximation (low-frequency) 
			and detail (high-frequency) coefficients across decomposition levels. 
			Artifacts, such as ocular blinks or EMG bursts, 
			manifest as sparse, high-amplitude coefficients that can be selectively 
			silenced using thresholds (f.e, hard or soft), 
			followed by reconstruction---thus removing noise while preserving neural transients like epileptic 
			spikes \cite{grobbelaar2022survey}. 

			\subsection{Evaluation metrics for EEG denoising in the CHB-MIT dataset}

			In order to evaluate the performance of 
			different wavelet-based filters for EEG denoising, 
			quantitative metrics have been computed over entire recordings 
			per channel and filter configuration. 
			Metrics used were:

			\begin{itemize}
			    \item \textbf{Signal-to-Noise Ratio (SNR, dB)}: Measure of the ratio between the 
				    power of the clean 
				    signal and the power of the noise. 
				    Higher values indicate better noise suppression 
				    while preserving the structure of the signal.
			    \item \textbf{Root Mean Square Error (RMSE, $\mu$V)}: This metric quantifies 
				        the average deviation between the denoised and reference signal in microvolts. 
					Lower values indicate closer similarity to the original signal.
			    \item \textbf{Normalized RMSE (NRMSE, \%)}: RMSE normalized by the dynamic range 
				        of the reference signal expressed as a percentage. 
					Lower values represent better performance.
			    \item \textbf{Correlation Coefficient}: Pearson's correlation among the denoised and 
				        reference signal, providing an assessment of similarity of the two waveforms.
					Observed values close to $1$ indicate high waveform preservation.
			    \item \textbf{Percent Root-mean-square Difference (PRD, \%)}: Quantification
				    of the relative distortion introduced by the denoising process. 
				    Lower values indicate less distortion.
			\end{itemize}

			For each one of the filters configuration, 
			metrics were computed 
			channel-wise and then averaged across 
			all channels in order to
			calculate their global performance score. 

			\subsection{Filter Selection Criteria}

			The selection of the optimal filter was based on a multi-criteria ranking strategy, where:
			\begin{enumerate}
			    \item Metrics where \textit{higher} values indicate better performance (\textbf{SNR}, \textbf{Correlation}) were ranked in descending order.
			    \item Metrics where \textit{lower} values indicate better performance (\textbf{RMSE}, \textbf{NRMSE}, \textbf{PRD}) were ranked in ascending order.
			    \item The ranks from all metrics were averaged to obtain an overall performance rank for each filter.
			\end{enumerate}

			The filter with the lowest average rank was considered the best compromise between 
			noise reduction and signal fidelity. According to this 
			evaluation, the \textbf{SYM8, level 4, threshold 0.5, hard thresholding} 
			filter presented the highest overall performance, exhibiting:
			\begin{itemize}
			    \item the highest SNR values,
			    \item one of the lowest RMSE and the lowest NRMSE value,
			    \item the highest correlation coefficient,
			    \item and the lowest PRD.
			\end{itemize}

			This indicates that the chosen filter effectively suppressed noise
			while preserving the morphological features of the EEG signal,
			making it the most suitable choice for subsequent analysis.
			The results of the benchmarking those metrics in 5 EDF 
			recording files which include seizures
			are presented in the following figure.

			\begin{figure}[h]
			    \centering
			    \includegraphics[width=1.1\textwidth]{eeg_metrics_result.png}
			    \caption{The top 10 filter configurations per EEG metric. RMSE values scaled.}
			    \label{}
			\end{figure}

			
			In the following figures we present a visualization using the recording named \textit{chb01\_03.edf}
			comparing the original 10 first EEG channels against the filtered ones with the respective wavelets filters.

			\begin{figure}[H]
			    \centering
			    \includegraphics[width=1.2\textwidth]{wav1.png}
			    \caption{}
			    \label{}
			\end{figure}


			\begin{figure}[H]
			    \centering
			    \includegraphics[width=1.2\textwidth]{wav2.png}
			    \caption{}
			    \label{}
			\end{figure}

\begin{comment}
			\begin{figure}[h]
			    \centering
			    \includegraphics[width=1.2\textwidth]{wav3.png}
			    \caption{}
			    \label{}
			\end{figure}


			\begin{figure}[h]
			    \centering
			    \includegraphics[width=1.2\textwidth]{wav4.png}
			    \caption{}
			    \label{}
			\end{figure}

			\begin{figure}[h]
			    \centering
			    \includegraphics[width=1.2\textwidth]{wav5.png}
			    \caption{}
			    \label{}
			\end{figure}
\end{comment}

	\newpage

		\section{Phase space reconstruction of EEG signals}

			In order to analyze multi-channel EEG's nonlinear dynamical system, the 
			reconstruction of the underlying phase space is required, 
			based on the scalar measurements of each channel. 
			According to Takens' embedding theorem \cite{takens1981}, 
			a time series $x(t)$ can be embedded 
			in an $m$-dimensional space using time-delay coordinates:

			\begin{equation}
			\vec{y}(t) = \left[x(t), x(t+\tau), x(t+2\tau), \ldots, x(t+(m-1)\tau)\right]
			\end{equation}

			where $m$ is the embedding dimension 
			and $\tau$ is the time delay. The critical challenge lies in determining 
			the appropriate values for these parameters to faithfully reconstruct the system's dynamics without distortion.
	

		\subsection{Determination of Embedding Parameters}
		The reconstruction of the phase space from a single time series 
		\( x(t) \) requires the specification of two parameters: 
		the time delay \( \tau \) and the embedding dimension \( m \). 
		These two parameters determine how the reconstruction will represent
		and how close it will reveal the underlying dynamics without distortion.

			\begin{figure}[H]
				    \centering
				    \includegraphics[width=0.8\linewidth]{ami.png}
				    \caption{Calculation of $\tau$ using AMI for a sample EEG channel. The first minimum of the AMI function (green dashed line) is chosen to become the optimal $\tau$ to ensure independence between delay coordinates.}
				    \label{fig:ami_plot}
			\end{figure}



		\subsubsection{Calculation of time delay \( \tau \) utilizing mutual information}
			The time delay \( \tau \) can be estimated by applying the 
			\textit{Average Mutual Information} (AMI) method, a concept which was first introduced 
			by Fraser and Swinney~\cite{fraser1986}. 
			In contrast to linear autocorrelation, 
			mutual information has the ability to capture both linear and nonlinear dependencies among
			the original time series \( x(t) \) and its delayed version \( x(t + \tau) \).

			The mutual information \( I(\tau) \) between \( x(t) \) and \( x(t + \tau) \) is defined as:
			\[
			I(\tau) = \sum_{x(t),\, x(t+\tau)} P(x(t), x(t+\tau)) \, \log_2 \left( \frac{P(x(t), x(t+\tau))}{P(x(t)) \, P(x(t+\tau))} \right)
			\]
			where \( P(\cdot) \) denotes probability.

			The optimal time delay \( \tau \) is chosen as the value at which 
			\( I(\tau) \) reaches its \textit{first minimum}. 
			This value indicates a good compromise 
			between independence (too small \( \tau \)) and irrelevance (too large \( \tau \)) 
			of the coordinates in the embedding vector.
				

		


			\subsubsection{Estimating embedding dimension \( m \) using false nearest neighbors approach}

			When the embedding dimension $m$ is too small, 
			the phase space becomes \emph{projected} rather than 
			properly \emph{embedded}. 
			This projection can create artificial neighborhoods where points appear to be close due to 
			geometrical constraints of the space rather than their actual dynamical similarity. 
			These are named as \emph{false nearest neighbors}. 
			An example of such an occurance can be observed in Figure \ref{fig:fnn_schematic}.

				\begin{figure}[h!]
				    \centering
				    \includegraphics[width=0.8\linewidth]{fnn_estim.png}
				    \caption{Calculation of the embedding dimension using the FNN scheme.}
				    \label{fig:fnn_plot}
				\end{figure}


			\begin{figure}[h]
			\centering
			\includegraphics[width=0.8\textwidth]{fnn_schematic.png}
			\caption{Schematic illustration of false neighbors. In insufficient embedding dimension (down), points A and B appear neighbors due to projection. When proper embedding is employed(up), their true separation is revealed.}
			\label{fig:fnn_schematic}
			\end{figure}

			Mathematically, two points $\vec{y}_i$ and $\vec{y}_j$ are false neighbors if their distance increases significantly when embedded in higher dimension:

			\begin{equation}
			\frac{\|\vec{y}_i^{(m+1)} - \vec{y}_j^{(m+1)}\|}{\|\vec{y}_i^{(m)} - \vec{y}_j^{(m)}\|} > R_{\text{tol}}
			\end{equation}

			where $R_{\text{tol}}$ is a tolerance threshold (typically 10--15).

			The False Nearest Neighbors(FNN) method \cite{kennel1992} 
			provides a systematic approach in order to determine the 
			minimal sufficient embedding dimension. The method's steps are:

			\begin{enumerate}
			\item For each point in dimension $m$, identify its nearest neighbor.
			\item Embed the data in dimension $m+1$.
			\item Calculation the relative distance increase between each point and its former neighbor.
			\item If the increase exceeds predetermined thresholds, the neighbor point is classified as a false neighbor
			\item The optimal $m$ is the smallest dimension where the fraction of 
				false neighbors drops below an acceptable level (typically 1--5\%)
			\end{enumerate}

			Both relative and absolute criteria are included in the algorithm:

			\begin{align}
			\text{Relative:} &\quad \frac{\|\vec{y}_i^{(m+1)} - \vec{y}_j^{(m+1)}\|}{\|\vec{y}_i^{(m)} - \vec{y}_j^{(m)}\|} > R_{\text{tol}} \\
			\text{Absolute:} &\quad \|\vec{y}_i^{(m+1)} - \vec{y}_j^{(m+1)}\| > A_{\text{tol}} \cdot \sigma_x
			\end{align}

			where $\sigma_x$ is the standard deviation of the time series.
			
			When optimal parameters $\tau$ and $m$ have been determined for a given signal, 
			the phase space can be reconstructed according to Takens' theorem. This reconstruction provides the
			geometric picture of the underlying dynamics.

			Figure \ref{fig:phase_space_3d} presents the reconstructed phase space for a normal EEG segment from channel 'Fp1-F7' from CHB-MIT 's chb24\_01.edf data.

				\begin{figure}[h!]
				    \centering
				    \includegraphics[width=0.7\linewidth]{phase_space_3d.png} % Assuming you make a 2-panel figure
				    \caption{3D phase space reconstruction for a 3-second EEG segment's channel.}
				    \label{fig:phase_space_3d}
				\end{figure}


			%%% NOTE: commented maybe not to write? 
			%Applying FNN to EEG data is particularly 
			%important for several reasons. EEG signals exhibit 
			%nonstationary characteristics, making fixed embedding parameters 
			%suboptimal and contain measurement noise and artifacts 
			%that can distort phase space reconstruction. 
			%In addition the optimal embedding may vary across subjects, 
			%brain states, and recording conditions and FNN provides a data-driven approach that adapts to individual recordings.


					
	\newpage

				\section{Recurrence Quantification Analysis (RQA)}

				Having reconstructed the phase space trajectory of the EEG signals, 
				the next step is to analyze its dynamical properties. 
				Recurrence Quantification Analysis is a powerful nonlinear method that 
				provides precisely this functionality by quantifying the number and duration 
				of recurrences of a dynamical system to its previous states \cite{theoryReviewRQA}. 
				The core of this quantification process is the \emph{Recurrence Plot (RP)}, 
				a visualization which denotes the times at which the phase space 
				trajectory revisits approximately the same area.
				In most non trivial cases, a phase space  does not have a dimension (two or three)
				which allows a direct visualization, so for higher dimensional phase spaces 
				the only solution is a projection into a two or three dimensional space. 
				However, RP enables the examination of a higher-dimensional phase space trajectory 
				via its two-dimensional representation of its recurrences.

				\subsection{The Recurrence Plot (RP)}

				RP is a symmetric, two-dimensional matrix that visualizes the recurrences of states.
				For a reconstructed trajectory \(\vec{y}(t)\) of length \(N\), 
				the recurrence matrix \(\mathbf{R}\) is defined as:

				\begin{equation}
				R_{i,j} = \Theta(\varepsilon - \|\vec{y}(i) - \vec{y}(j)\|), \quad i,j = 1, \ldots, N
				\end{equation}

				where:
				\begin{itemize}
				    \item \(\Theta(\cdot)\) is the Heaviside step function (\(\Theta(x)=0\) if \(x<0\), and \(\Theta(x)=1\) otherwise),
				    \item \(\varepsilon\) is a predefined distance threshold (radius),
				    \item \(\|\cdot\|\) is a norm.
				\end{itemize}
	
				By interpreting the RP, several metrics can be extracted for further analysis.
				
				\subsection{Key RQA Metrics and Their Interpretation}
					\label{subsec:rqa_metrics}

					RQA provides a set of metrics that can quantify the 
					number and the duration of the recurrences of a dynamical system.
					These metrics are categorized by those which are based on diagonal structures, 
					which relate to the predictability and deterministic nature of the system, 
					and those that are based on vertical structures, which can capture laminar 
					states or chaos-chaos transitions.

					The definitions of the core RQA metrics, as implemented in tools like the utilized \texttt{PyRQA}, 
					are as follows \cite{marwan_website}:

					\begin{description}

					\item[Recurrence Rate (\textbf{RR})]
					The recurrence rate is the simplest measure, defined as the density of recurrence points in the RP. It corresponds to the probability that a state recurs and is analogous to the correlation sum.
					\[
					RR = \frac{1}{N^2} \sum_{i,j=1}^{N} R_{i,j}
					\]

					\item[Determinism (\textbf{DET})]
					Determinism quantifies the percentage of recurrence points that form diagonal lines. Diagonal lines are a signature of deterministic dynamics, where segments of the trajectory run in parallel for some time. A higher DET indicates a more predictable, deterministic system.
					\[
					DET = \frac{\sum_{l=l_{\text{min}}}^{N} l \, P(l)}{\sum_{l=1}^{N} l \, P(l)}
					\]
					where \( P(l) \) is the histogram of diagonal line lengths \( l \), and \( l_{\text{min}} \) is the minimum line length (typically 2).

					\item[Laminarity (\textbf{LAM})]
					Laminarity measures the percentage of recurrence points that form vertical lines. Vertical lines indicate states that do not change or change very slowly for a period (laminar states). It can detect chaos-chaos transitions or intermittency.
					\[
					LAM = \frac{\sum_{v=v_{\text{min}}}^{N} v \, P(v)}{\sum_{v=1}^{N} v \, P(v)}
					\]
					where \( P(v) \) is the histogram of vertical line lengths \( v \), and \( v_{\text{min}} \) is the minimum line length.

					\item[Ratio (\textbf{RATIO})]
					The ratio is a measure of complexity, calculated as the ratio between DET and RR. It can be sensitive to transitions between order and chaos.
					\[
					RATIO = \frac{N^2 \sum_{l=l_{\text{min}}}^{N} l \, P(l)}{\left( \sum_{l=1}^{N} l \, P(l) \right)^2}
					\]

					\item[Average Diagonal Line Length (\textbf{L})]
					This metric represents the average time that two segments of the trajectory remain close, providing an estimate of the mean prediction time.
					\[
					L = \frac{\sum_{l=l_{\text{min}}}^{N} l \, P(l)}{\sum_{l=l_{\text{min}}}^{N} P(l)}
					\]

					\item[Trapping Time (\textbf{TT})]
					Trapping time is the average length of vertical lines, quantifying the mean time the system remains trapped in a specific state (laminarity in time).
					\[
					TT = \frac{\sum_{v=v_{\text{min}}}^{N} v \, P(v)}{\sum_{v=v_{\text{min}}}^{N} P(v)}
					\]

					\item[Longest Diagonal Line (\textbf{L$_{\text{max}}$})]
					The length of the longest diagonal line in the RP is related to the Lyapunov exponent of the system. A shorter \( L_{\text{max}} \) suggests a faster divergence of trajectories, which is a hallmark of chaos.
					\[
					L_{\text{max}} = \max(\{l_i \; | \; i=1,\ldots,N_l\})
					\]

					\item[Divergence (\textbf{DIV})]
					Divergence is the inverse of \( L_{\text{max}} \). It is related to the Kolmogorov-Sinai entropy and the sum of the positive Lyapunov exponents, providing a measure of how quickly nearby trajectories diverge.
					\[
					DIV = \frac{1}{L_{\text{max}}}
					\]

					\item[Longest Vertical Line (\textbf{V$_{\text{max}}$})]
					The length of the longest vertical line is another indicator of the system's laminar behavior.
					\[
					V_{\text{max}} = \max(\{v_i \; | \; i=1,\ldots,N_v\})
					\]

					\item[Entropy (\textbf{ENTR})]
					The Shannon entropy of the probability distribution \( p(l) \) of the diagonal line lengths. It reflects the complexity of the deterministic structure in the system. A higher ENTR indicates a more complex and less periodic dynamics.
					\[
					ENTR = - \sum_{l=l_{\text{min}}}^{N} p(l) \ln p(l), \quad \text{where } p(l) = \frac{P(l)}{\sum_{l=l_{\text{min}}}^{N} P(l)}
					\]

					\item[Trend (\textbf{TREND})]
					Trend quantifies the paling of the RP towards its edges, which can be caused by non-stationarity in the data (e.g., a slow drift in the mean of the signal). It is calculated as the slope of the linear regression of the local recurrence rate \( RR_i \) over the distance from the main diagonal.
					\[
					TREND = \frac{\sum_{i=1}^{\tilde{N}} (i - \tilde{N}/2)(RR_i - \langle RR_i \rangle)}{\sum_{i=1}^{\tilde{N}} (i - \tilde{N}/2)^2}
					\]
					where \( \tilde{N} \) is the number of diagonals parallel to the Line of Identity (LOI) that are considered, and \( RR_i \) is the recurrence rate in the \( i \)-th diagonal.

					\end{description}

					These metrics, when applied to EEG signals, allow for the characterization of
					the brain's dynamic states. For example, in an epileptic seizures occurance, often 
					higher determinism (DET), laminarity (LAM) or recurrence rate(RR) is observed
					compared to the more stochastic and complex inter-ictal states.
					This fact makes RQA metrics a viable solution for identifying pathological patterns.	


				\subsection{Cross-Recurrence Quantification Analysis (CRQA)}
					\label{subsec:crqa_theory}

					While Recurrence Quantification Analysis (RQA) is powerful for analyzing the dynamics of a single system, 
					many real-world phenomena, including brain activity, involve the interaction between multiple subsystems.
					Cross-Recurrence Quantification Analysis (CRQA) extends the concepts of RQA to analyze the coupling, synchronization
					and degree of similarity that the dynamics between two different systems present\cite{marwan2007}.

					\subsubsection{The Cross-Recurrence Plot (CRP)}

					The foundation of CRQA is the Cross-Recurrence Plot (CRP). 
					For two reconstructed phase space trajectories \( \vec{x}(i) \) from system \( X \) and \( \vec{y}(j) \) 
					from system \( Y \), both of length \( N \), the cross-recurrence matrix is defined as:

					\begin{equation}
					CR_{i,j} = \Theta(\varepsilon - \|\vec{x}(i) - \vec{y}(j)\|), \quad i,j = 1, \ldots, N
					\end{equation}

					Unlike the standard RP, which is symmetric about the main diagonal (Line of Identity, LOI), 
					the CRP is generally \emph{not symmetric}. This asymmetry can reveal directional relationships 
					or leader-follower dynamics alonside the two systems.

					\subsubsection{Interpretation of CRQA Metrics}

					The same quantitative measures defined for RQA (Section~\ref{subsec:rqa_metrics}) can be applied to the CRP, but their interpretation shifts from describing \emph{self-similarity} to describing \emph{coupling} and \emph{interaction}:

					\begin{itemize}
					    \item \textbf{Cross-Recurrence Rate (CRR)}: The probability that the state of system \( X \) at time \( i \) is close to the state of system \( Y \) at time \( j \). A high CRR indicates overall similar states between the two systems.
					    
					    \item \textbf{Cross-Determinism (CDET)}: The percentage of recurrent points in the CRP that form diagonal lines. Diagonal lines occur when the two systems follow a similar path in phase space for some time. \textbf{This is a crucial metric for epilepsy detection}, as it quantifies the transient synchronization between different brain regions. A seizure often manifests as increased CDET between channels in the epileptogenic zone.
					    
					    \item \textbf{Cross-Laminarity (CLAM)}: Measures the laminarity between the two systems, indicating when one system gets trapped in a state while the other changes.
					    
					    \item \textbf{Average Diagonal Line Length (L)} in the CRP estimates the mean time that the two systems remain synchronized or follow a similar trajectory.
					\end{itemize}

					\subsubsection{Why CRQA for Multi-Channel EEG?}

					Applying CRQA to pairs of EEG channels is particularly well-suited for epilepsy detection for several reasons:

					\begin{itemize}
					    \item \textbf{Synchronization Detection}: Epileptic seizures are characterized by abnormal, excessive synchronization of neuronal populations. CRQA directly quantifies this synchronization in the phase space.
					    \item \textbf{Nonlinear and Non-stationary}: CRQA does not assume linearity or stationarity, making it robust for analyzing the complex, transient dynamics of EEG signals.
					    \item \textbf{Directional Insights}: While not explored in all analyses, the potential asymmetry of the CRP can, in principle, help identify the propagation path of a seizure.
					    \item \textbf{Focus on Interaction}: It moves beyond analyzing individual channels in isolation to directly measure the dynamic interplay between different brain regions, which is often where the pathology lies.
					\end{itemize}

					In this thesis, CRQA is employed to compute a set of features (Table~\ref{tab:pyrqa_metrics}) for all unique pairs of EEG channels. These features capture the complex synchronization patterns that distinguish pre-ictal, ictal, and inter-ictal states, forming the basis for the subsequent machine learning classification.





	\newpage

					
				\section{Methodology}
				\label{sec:methodology}

				In this section the methodology for processing EEG data is described in order to perform CRQA to analyze epileptic and non-epileptic brain activity. 
				The approach consists of different parts such as loading and segmenting EEG recordings, 
				extracting non-overlapping time windows, selecting the embedding parameters and computing CRQA features for channel pairs. 
				The methodology is implemented in Python using libraries such as \texttt{numpy}, \texttt{torch}, \texttt{pyopencl}, and \texttt{pyrqa} \cite{pyrqacitation}.

				\subsection{Data preprocessing and windowing}
				\label{subsec:data_preprocessing}

				Prior filtererd EEG recordings are stored in NumPy array format (\texttt{.npy}), 
				accompanied by metadata specifying the sampling frequency (\(f_s\)) and channel information 
				(22 channels with their respective labels as \texttt{FP1-F7}, \texttt{F7-T7}, ..., \texttt{FT10-T8}). 
				The time axis is computed as \(t = \frac{n}{f_s}\), where \(n\) is the sample index and \(f_s\) is the sampling frequency in Hertz (Hz).

				Each EEG channel is segmented into continuous regions based on predefined boundaries from CHB-MIT dataset \cite{chbmitDataset} annotations, distinguishing epileptic from non-epileptic segments. 
				Then, each segment is further divided into non-overlapping time windows of fixed size (512 samples, equivalent to 2 seconds at 256 Hz). 
				For each segment, the number of windows is calculated by performing integer division of the segment length by the window size and discarding any incomplete windows. 
				Each window is associated with a segment index, window index within the segment, start and end sample indices, and a label (1 for epileptic, 0 for non-epileptic).

				\subsection{Embedding parameters selection}
				\label{subsec:embedding_parameters}
				
				The embedding dimension \texttt{m} and time delay $\tau$ were set to 3 and 1, respectively, 
				following common practice in EEG analysis. This choice was also motivated by known limitations that the FNN algorithm experiences 
				when applied to noisy and autocorrelated signals, such as EEG. 
				For instance, \cite{FredkinRiceFNN} have shown that FNN can falsely indicate low-dimensional determinism in 
				autocorrelated stochastic processes, while \cite{RhodesMorariFNN} showed that the effects of noise can actually
				lead in overestimation of the embedding dimension. 
				In order to mitigate these effects and keep a consist methodology across a 
				large dataset, the presented methodology adopts fixed values for the embedding parameters rather than optimizing them
				per recording or window.
				The decision to use constant values is influenced by applied precedents in the EEG literature. 
				McSharry et al.\cite{mcsharry} have applied with success fixed embedding parameters in their multi-channel scalp EEG seizure research, 
				arguing that nonlinear methods must justify their complexity over simpler linear benchmarks. 
				In our case, fixed parameters ensure consistency across the large CHB-MIT dataset and help avoid overfitting to local dynamics that may not generalize.
				
				\subsection{Threshold selection}
				Radius fraction R is utilized for determination of the percentage of mean diameter of the 
				reconstructed phase space where a recurrence can occur.
				In order to estimate and standardize R,
				an exploration of its effect on CRPs and RR/DET metrics is performed. 
				By keeping constant $\tau = 1$ and \textbf{\textit{m}} = 3
				CRPs are generated by selecting random patients and random recording windows of the dataset,
				while computing the mean channel-wise recurrence rate and mean channel-wise determinism 
				from the 22x22 CRPs features.

				Results of RR and DET are presented in the following table.
				The different explored values for radius fraction are set to be 
				{ 0.1, 0.15, 0.20 and 0.30} for this experiment.
								
				\begin{table}[h!]
				\centering
				\caption{Comparison of RR and DET values for different radius (R) values}
				\begin{tabular}{l S[table-format=1.2] S[table-format=2.2] S[table-format=2.2] l}
				\toprule
				\textbf{Recording} & \textbf{R} & \textbf{RR (\%)} & \textbf{DET (\%)} & \textbf{Window} \\
				\midrule
				patient\_24 & 0.10 & 9 & 77.36 & Normal \\
				patient\_24 & 0.15 & 16.7 & 85.3 & Normal \\
				patient\_24 & 0.20 & 23.8 & 89 & Normal \\
				patient\_24 & 0.30 & 36.4 & 93.25 & Normal \\
				patient\_24 & 0.10 & 16.35 & 97.6 & Epileptic \\
				patient\_24 & 0.15 & 26 & 99 & Epileptic \\
				patient\_24 & 0.20 & 35.16 & 99.44 & Epileptic \\
				patient\_24 & 0.30 & 51.9 & 99.74 & Epileptic \\
				\midrule
				patient\_10 & 0.10 & 10.6 & 83.5 & Normal \\
				patient\_10 & 0.15 & 16.54 & 86.8 & Normal \\
				patient\_10 & 0.20 & 22.13 & 88.97 & Normal \\
				patient\_10 & 0.30 & 32.5 & 92.87 & Normal \\
				patient\_10 & 0.10 & 14.94 & 93.6 & Epileptic \\
				patient\_10 & 0.15 & 22.9 & 95.7 & Epileptic \\
				patient\_10 & 0.20 & 30.3 & 96.41 & Epileptic \\
				patient\_10 & 0.30 & 43.59 & 97.16 & Epileptic \\
				\midrule
				patient\_8 & 0.10 & 7.08 & 59.26 & Normal \\
				patient\_8 & 0.15 & 11.82 & 65,12 & Normal \\
				patient\_8 & 0.20 & 16.33 & 67.59 & Normal \\
				patient\_8 & 0.30 & 24.76 & 71.73 & Normal \\
				patient\_8 & 0.10 & 16.4 & 98.3 & Epileptic \\
				patient\_8 & 0.15 & 25 & 99.46 & Epileptic \\
				patient\_8 & 0.20 & 33.1 & 99.60 & Epileptic \\
				patient\_8 & 0.30 & 47.71 & 99.76 & Epileptic \\
				\midrule
				patient\_1 & 0.10 & 18.31 & 96.02 & Normal \\
				patient\_1 & 0.15 & 27.83 & 97.70 & Normal \\
				patient\_1 & 0.20 & 36.61 & 97.89 & Normal \\
				patient\_1 & 0.30 & 52.08 & 98.89 & Normal \\
				patient\_1 & 0.10 & 20.89 & 96.11 & Epileptic \\
				patient\_1 & 0.15 & 33.93 & 98.46 & Epileptic \\
				patient\_1 & 0.20 & 45.61 & 99.22 & Epileptic \\
				patient\_1 & 0.30 & 64.64 & 99.68 & Epileptic \\

				\bottomrule
				\end{tabular}
				\end{table}
				
				As it can be observed, both RR and DET increase while the radius fraction $R$ increases
				for all patients/windows combinations, since by having a larger radius there are more points to be considered as recurrent in the phase space.
				Epileptic windows present consistently higher RR and DET values when compared with normal windows at the same $R$ and same patients.
				It should be noted that there is inter-patient variability also, 
				suggesting that optimal radius selection may benefit from patient-specific tuning.
				Additionally, DET values in epileptic windows approach saturation near 100\% as $R$ increases, a fact that suggests 
				having a moderate radius fraction (e.g., $R = 0.15$--$0.20$) could provide a 
				better balance between sensitivity and specificity in CRP analysis for the determinism metric.


				\begin{figure}[htbp]
				    \centering
				    \begin{subfigure}[t]{0.45\textwidth}
					\centering
					\includegraphics[width=\textwidth]{rr_compare/r01_epil.png}
					\caption{radius fraction = 0.1}
					\label{subfig:rp1}
				    \end{subfigure}
				    \hfill % Adds horizontal space between subfigures
				    \begin{subfigure}[t]{0.45\textwidth}
					\centering
					\includegraphics[width=\textwidth]{rr_compare/r015_epil.png}
					\caption{radius fraction = 0.15}
					\label{subfig:rp2}
				    \end{subfigure}

				    \vspace{0.5cm} % Adds vertical space between rows

				    \begin{subfigure}[t]{0.45\textwidth}
					\centering
					\includegraphics[width=\textwidth]{rr_compare/r02_epil.png}
					\caption{radius fraction = 0.2}
					\label{subfig:rp3}
				    \end{subfigure}
				    \hfill
				    \begin{subfigure}[t]{0.45\textwidth}
					\centering
					\includegraphics[width=\textwidth]{rr_compare/r03_epil.png}
					\caption{radius fraction = 0.3}
					\label{subfig:rp4}
				    \end{subfigure}

				    \caption{CRPs for the selected EEG channel pairs, for an epileptic window. (a) R = 0.1 (b) R = 0.15 (c) R = 0.2 (d) R = 0.3}
				    \label{fig:rp_grid}
				\end{figure}


				\subsection{Cross Recurrence Quantification Analysis (CRQA)}
				\label{subsec:crqa}

				CRQA quantifies the recurrent patterns between pairs of EEG channels within each time window. 
				The \texttt{pyrqa} library is used with OpenCL acceleration for efficient computation. 
				The process is as follows:

				\begin{enumerate}
				    \item \textbf{Time Series Length Validation}: For each window pair, the lenght's of the two time series 
					    are compared on having same length to ensure compatibility.
				    \item \textbf{Phase Space Reconstruction}: The time series are embedded into a phase space using the 
					    optimal \(\tau\) and \(m\), via the \texttt{TimeSeries} class in \texttt{pyrqa}.
				    \item \textbf{Radius Selection}: The radius for defining recurrence is computed using the 
					    Phase Space Separation (PSS) method (\texttt{pss} function). The maximum distances in the phase spaces 
						of both channels are averaged to obtain a mean diameter, and the radius is set 
						to 15\% of this value (\texttt{radius\_fraction=0.15}).
				    \item \textbf{CRQA Computation}: The \texttt{RQAComputation} class constructs a cross-recurrence matrix 
					    using a \texttt{FixedRadius} neighborhood, Euclidean metric, and Theiler corrector of 1. 
						The computation yields 16 CRQA features, listed in Table~\ref{tab:pyrqa_metrics}, plus the segment label as the 17th feature.
				\end{enumerate}

				\begin{table}[h]
				\centering
				\caption{Quantitative measures computed by PyRQA}
				\label{tab:pyrqa_metrics}
				\begin{tabular}{ll}
				\toprule
				\textbf{Metric} & \textbf{Abbreviation} \\
				\midrule
				Recurrence Rate & RR \\
				Determinism & DET \\
				Average Diagonal Line Length & \(L_{\text{avg}}\) \\
				Longest Diagonal Line Length & \(L_{\text{max}}\) \\
				Divergence & DIV \\
				Entropy Diagonal Lines & \(H_{\text{diag}}\) \\
				Laminarity & LAM \\
				Trapping Time & TT \\
				Longest Vertical Line Length & \(V_{\text{max}}\) \\
				Average White Vertical Line Length & \(W_{\text{avg}}\) \\
				Longest White Vertical Line Length & \(W_{\text{max}}\) \\
				Longest White Vertical Line Divergence & \(W_{\text{max}}^{-1}\) \\
				Entropy Vertical Lines & \(H_{\text{vert}}\) \\
				Entropy White Vertical Lines & \(H_{\text{wvert}}\) \\
				Ratio of Determinism to Recurrence Rate & DET/RR \\
				Ratio of Laminarity to Determinism & LAM/DET \\
				\bottomrule
				\end{tabular}
				\end{table}


				\subsection{Algorithm Summary}
				\label{subsec:algorithm}

				The methodology is summarized in Algorithm~\ref{alg:crqa_computation}, which outlines the CRQA computation for each window and channel pair.

				\begin{algorithm}
				\caption{Cross Recurrence Quantification Analysis (CRQA) for EEG Windows}
				\label{alg:crqa_computation}
				\begin{algorithmic}[1]
				\State \textbf{Input}: EEG windows \( \{X_{c,w}\} \) for channels \( c \in C \), windows \( w = 1, \dots, N_w \), 
					where \( N_w = \text{number of windows} \), number of electrodes \( N_e \)
				\State \textbf{Output}: CRQA feature matrix \( M \) of shape \( (N_w, N_e, N_e, 17) \)

				\For{each window index \( w = 1 \) to \( N_w \)}
				    \For{each channel pair \( (c_1, c_2) \in C \times C \)}
					\State \textbf{Time Series Preparation}
					\State Ensure the input time series \( X_{c_1,w} \) and \( X_{c_2,w} \) have the same length
					\State \textbf{Set embedding parameters}
					\State Set \( \tau = 1 \)
					\State Set \( m = 3 \)
					\State \textbf{CRQA Computation}
					\State Construct cross-recurrence plot for \( (X_{c_1,w}, X_{c_2,w}) \) using \texttt{PyRQA} with:
					\State \quad Fixed radius neighborhood \( r = 0.15 \times \text{mean diameter} \), Euclidean metric, Theiler corrector = 1
					\State Extract 16 CRQA features: \{RR, DET, \( L_{\text{avg}} \), \( L_{\text{max}} \), DIV, \( H_{\text{diag}} \), LAM, 
					TT, \( V_{\text{max}} \), \( W_{\text{avg}} \), \( W_{\text{max}} \), 
					\( W_{\text{max}}^{-1} \), \( H_{\text{vert}} \), \( H_{\text{wvert}} \), DET/RR, LAM/DET\}
					\State \textbf{Feature Storage}
					\State Append window label \( l_w \) to features
					\State Store features in \( M[w, c_1, c_2, :] \)
				    \EndFor
				\EndFor

				\State \textbf{Return} RQA feature matrix \( M \)
				\end{algorithmic}
				\end{algorithm}

				 \subsection{Feature aggregation}
				 \label{subsec:feature_aggregation}
				 The resulting RQA feature matrix has dimensions \([N_w, N_e, N_e, 17]\), where \(N_w\) is the number of windows, 
				\(N_e\) is the number of channels, and 17 represents the 16 CRQA features plus the segment label(epileptic or normal). 
				 To summarize CRQA features across all channel pairs for each window, 
				 a mean feature matrix is computed by averaging the 16 CRQA features across all channel pairs, 
				 resulting in a final matrix (\texttt{mean\_\_feature\_matrix}) of shape \([N_w, 17]\), where the last column retains the window label. 
				 This matrix summarizes the average dynamical interactions within each window.

				 It should be noted that prior to the averaging all feature metrics that presented invalid numerical values(such as \textit{nan} or \textit{inf})
				 were set to 0. This can occur each time where
				 the reccurence rate is 0 or a value very close to zero, having a consequence to propagate invalid values 
				 in the set of \{Determinism, Average Diagonal Line Length, Divergence, Laminarity, Trapping Time, DET/RR and LAM/DET\} features computed by the PyRQA computation.

\begin{comment}	
			\subsection{Post-processing and Merging of CRQA Feature Segments}

			After extracting all of the CRQA features for each 2-second EEG 
			segment and the averaging across all 22~$\times$~22 electrode pairs, 
			every segment holds a represented by a 16-dimensional feature vector plus its dedicated label. 
			At this stage, the dataset contains a disproportionately large number of 
			normal (non-epileptic) segments in comparisson to the epileptic ones, 
			a fact that leads into strong class imbalance and increased variance 
			within the normal class.

			To address this issue, an additional post-processing step was applied in which consecutive normal segments were merged into fixed-size groups of \(k\) (\(k = 10\)). Each group of \(k\) normal segments 
				\[
				\mathbf{x}_1, \mathbf{x}_2, \ldots, \mathbf{x}_k \in \mathbb{R}^{16}
				\]
				was replaced by a single representative feature vector obtained as their arithmetic mean:
				\begin{equation}
				    \bar{\mathbf{x}} = \frac{1}{k} \sum_{i=1}^{k} \mathbf{x}_i.
				\end{equation}

				Epileptic segments were \emph{not} merged. This choice is justified by the highly dynamic and non-stationary nature of seizure activity, for which temporal averaging could obscure important patterns or abrupt state transitions that are meaningful for classification.

				The output of this merging procedure is a new feature matrix in which:
				\begin{itemize}
				    \item normal EEG activity is represented by smoothed, low-variance averaged feature vectors,
				    \item epileptic activity is retained at its original 2-second resolution, and
				    \item all resulting samples remain homogeneous 16-dimensional CRQA feature vectors.
				\end{itemize}

				Each merged or non-merged sample is appended with the corresponding class label \( y \in \{0,1\} \), 
				resulting in a final dataset of the form:
				\[
				\mathbf{z}_j = [f_1, f_2, \ldots, f_{16}, y], \qquad j = 1, \ldots, M,
				\]
				where \(M\) depends on the number of merged normal blocks and retained epileptic segments.  
				This representation improves class balance, reduces the intra-class variability among normal segments, and preserves the discriminatory structure necessary for the subsequent machine-learning-based seizure detection.
\end{comment}



\begin{comment}
				\subsection{Synthetic Feature Engineering from Cross-Recurrence Quantification Analysis}
				\label{subsec:synthetic_features}

				In addition to the original 16 CRQA features of the \texttt{mean\_\_feature\_matrix}, that capture 
				fundamental aspects of nonlinear coupling between EEG channels, in order to achive 
				better discriminative power in our classification,
				10 additional synthetic features are engineered as combinations of the original CRQA metrics per example.

				The first group comprises four \textit{dynamically interpretable ratios}:
				\begin{itemize}
				\item \textbf{Determinism Efficiency (DET\_EFF)}: $\frac{\text{DET}}{\text{RR} + \varepsilon}$, quantifying the proportion of 
					recurrent points that form deterministic (diagonal) structures. During seizures, RR 
					increases due to hypersynchrony, but DET increases more selectively; this ratio captures the relative 
					strength of deterministic coupling.
				    \item \textbf{Diagonal Complexity Index (DIAG\_COMP)}: $\frac{\text{L\_entr}}{\log(\text{L\_max} + 1) + \varepsilon}$, a 
					scale-invariant measure of complexity in diagonal line distributions. Seizures typically reduce dynamical 
					complexity, lowering this index.
				    \item \textbf{Vertical-to-Diagonal Ratio (V\_D\_RATIO)}: $\frac{\text{LAM}}{\text{DET} + \varepsilon}$, reflecting 
					the balance between laminar (vertical) trapping and predictable (diagonal) trajectories. Elevated ratios 
					indicate intermittent ictal states.
				    \item \textbf{Recurrence Stability Index (REC\_STAB)}: $\frac{\text{L}}{\text{L\_max} + \varepsilon}$, measuring 
					homogeneity of predictability duration. Values near 1 signify sustained rhythmicity, characteristic of generalized seizures.
				\end{itemize}

				The second group includes six \textit{targeted interactions and transforms}:
				\begin{itemize}
				\item \textbf{RR\_x\_DET} and \textbf{DET\_x\_L}: Products capturing joint recurrence-determinism and deterministic predictability strength.
				    \item \textbf{LAM\_x\_TT}: Product of laminarity and trapping time, representing vertical structure persistence.
				    \item \textbf{L\_entr\_div\_L}: Normalized diagonal entropy, improving robustness to scale differences.
				    \item \textbf{log\_RR} and \textbf{log\_L\_max}: Log-transformed features to stabilize skewed distributions common in recurrence measures.

				\end{itemize}

				Here, $\varepsilon = 10^{-8}$ is utilized to prevent division by zero. 
				Synthetic engineered features are dimensionless and 
				bounded, ensuring numerical stability. 
				This approach avoids blind polynomial expansion in 
				favor of \textit{physiologically grounded combinations}, 
				aligning with best practices in nonlinear EEG biomarker design \cite{faes2014information, zhang2020cross}.
				
\end{comment}


				\section{Class Imbalance in RQA-EEG Data}

				Real world EEG datasets for epileptic seizure detection are by their nature imbalanced, 
				reflecting the transient nature of seizure's occurence in contrast to normal brain activity. 
				In our aggregated dataset, derived from CRQA features across the multi-channel EEG recordings,
				the distribution is stark: 98.99\% normal segments (89633 examples) against 1.01\% epileptic (901 examples), 
				yielding a $\sim$99:1 ratio. This skew is a well known challenge in machine 
				learning which is often reffered as the "imbalanced learning problem".

				As discussed by He and Garcia\cite{he2009}, standard classifiers, (f.e, Random Forest), 
				can achieve misleadingly high accuracy ($\sim$99\%) 
				by over-predicting the majority class, resulting in poor recall for epileptic events.
				In RQA based models, this bias can minimize the effects of discriminative patterns, 
				such as the elevated determinism (DET) or laminarity (LAM) in epileptic signals with 
				false negatives posing ethical risks, delaying interventions, while naive oversampling 
				(f.e, duplication) incorporates the risk of overfitting in minority RQA features.

				In order to mitigate this, data-level resampling strategies can be employed, focusing on 
				synthetic generation to enrich the minority class without discarding information gained from
				the majority class. 

				\begin{table}[h]
				\centering
				\caption{Class distribution in the preprocessed RQA-EEG dataset, highlighting severe imbalance.}
				\label{tab:imbalance}
				\begin{tabular}{lcc}
				\toprule
				Class & Percentage (\%) & Count \\
				\midrule
				Normal (0) & 98.99 & 89,633\\
				Epileptic (1) & 1.01 & 901 \\
				\bottomrule
				\end{tabular}
				\end{table}
				
				
				To address the higly imbalanced dataset, 
				three main methods can be employed.

				Regarding the data level, methods on creating more synthetic
				samples belonging to the minority class exist, such as
				Synthetic Minority Over-sampling Technique (SMOTE).
				With this method it is possible to generate synthetic 
				epileptic samples
				by interpolating between minority instances 
				and their 
				k-nearest neighbors in 
				feature space \cite{SMOTEref}. 
				Unlike random duplication, 
				SMOTE promotes diversity, reducing overfitting risks in 
				low-sample regimes. 

				On the classification algorithm level, specific weights and costs 
				can be used in order to favor the minority class.
				For instance, many classifiers, 
				such as SVM or tree-based models, allow for the assignment 
				of class weights. 
				These weights are usually set to be inversely proportional
				to the class frequencies, 
				pushing the model towards paying more attention on
				errors made on the minority class on the training phase. 
				In the same manner, cost-sensitive learning 
				methods define a higher penalty for misclassifying 
				minority class samples, optimizing for a cost function 
				that reflects the real-world imbalance.

				As a third alternative, ensemble methods leverage two or more
				base models to swift the skew of bias towards the majority class.
				An ensemble can integrate different classifiers and employ different
				aggregation strategies for its final classification decision, such as 
				majority vote, weighted votes or stacking.

\begin{comment}

				\section{Patient-Specific EEG Classification in Epilepsy}
				\label{sec:need-patient-specific}

				The development of robust and accurate EEG-based seizure detection 
				solutions challenging task due to the high 
				variability in EEG signals across different patients. 
				While a universal classifier trained on data from multiple subjects 
				would be the desirable goal for generalization, 
				there are evidence from studies strongly 
				suggesting that patient-specific 
				models offer superior performance and can better suit 
				for real-world clinical and wearable applications.

				Hussein et al.\cite{hussein} conducted a quantitative 
				analysis of intracranial EEG (iEEG) 
				data and concluded that statistical 
				characteristics 
				of both interictal and preictal EEG vary significantly between patients. 
				For instance, researchers observed that 
				the overall range and interquartile 
				range of iEEG sensor readings differed not only between 
				interictal and preictal 
				states but also across patients. 
				They state that \textit{“there is no typical trend in either interictal or preictal 
				data across different epileptic patients”} and that 
				\textit{“the iEEG data of each patient has its own characteristics 
				and its statistical features can solely be meaningful for building a 
				seizure prediction system for this particular patient”}. 
				This kind of inter-patient variability can limit the
				effectiveness of deploying a global model and underline 
				the necessity of 
				patient-specific approaches.

				Similarly, Qiu et al.\cite{lightseizurenet} highlighted 
				that \textit{“the features of the EEG signals 
				vary among different patients”} and that 
				\textit{“even for the same patient, the EEG signals could 
				change substantially when the patient is 
				at different modes, status (awake or asleep), or ages”}.
				Authors introduced a lightweight 
				deep learning model, called LightSeizureNet, 
				which was evaluated in both patient-independent 
				and patient-specific settings. 
				The patient-specific model achieved higher accuracy 
				than the patient-independent version, achieving 99.77\%. 
				In addition, patient-specific models 
				seem to be more computationally efficient 
				for real-time application, such as wearable systems, because of their ability
				to "personalize". %%maybe personalized medicine?
				Furthermore, authors reported that their 
				patient-specific model required only 3.7 million 
				multiply-accumulate operations (MACs), 
				compared to 6.2 million MACs for the patient-independent model, making it a more suitable solution for 
				resource constrained application domains
				such as a wearable/implantable device.
				
				Further supporting evidence can be found in a recent study by Vijay et al.\cite{vijay2025supervised}, 
				in which authors perform a systematic comparisson of five supervised machine learning models
				for both patient-specific and non-patient-specific seizure detection using the CHB-MIT scalp EEG database. 
				Their results demonstrate a clear performance advantage when patient-specific models are used, 
				because of their ability of understanding and learning from personalized seizure patterns. 
	
				\subsection{Patient specific channel rejection}
					%%full rewrite
				In the pursuit of efficient and robust seizure detection systems, recent advancements emphasize 
				patient-specific EEG channel selection to mitigate noise and redundancy inherent 
				in multi-channel recordings, particularly for wearable applications. 
				For instance, a 2025 study by Ferrara et al\cite{channelDropFerrara},
				introduces a PCA-based Channel Weight Coefficient (CWC) method that dynamically ranks
				and selects just two uncorrelated bipolar channels per patient from the standard
				21-electrode 10-20 system, leveraging temporal shifts in neural activity between inter-ictal and ictal phases.


\end{comment}





			\newpage


\bibliographystyle{plain}
\bibliography{references}

\end{document}
